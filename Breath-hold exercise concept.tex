\documentclass[12pt, letterpaper]{article}


\begin{document}

\title{\textbf{ DOES GENDER AFFECT HOW LONG ONE CAN HOLD THEIR
	BREATH.}}
\author{15/U/8383/EVE \\ MUSIIMEMARIA EUGENIA }
 \date{}
\maketitle 
\section{\textbf{INTRODUCTION} } 
Men will have a higher or lower lung volume than girls. The time
 one can take to hold their breath is based on indivdual size, gender, health,
age, ethnicity and geographical location. These are factors tht cause a
variation. Likewise taller people have a larger lung volume than short people.
Being athletic and in good shape also increases lung volume while health
conditions and the lack of exercise can reduce it. The time
one can take to hold their breat is mainly based on their lung volume.

\section{\textbf{PROBLEM STATEMENT} } 
 A recent idea led to the need to do exploratory research
to findout if there existed a gender pattern that influnces how long an individual
can hold their breath.

\section{\textbf{OBJECTIVES} } 
\textbf{-}One of the goals is to create a hypothesis for which gendercan hold their breath longest.\\
\textbf{-} Know how long one can hold their breath.\\
\textbf{-} Find out whether gender affects how long one can hold their breath.

\section{\textbf{METHODOLOGY} } 
\textbf{-} In this exercise, i will compare the time one can hold their breath with regard to the difference between genders while eliminating the differences caused by the previously mentioned variables.\\ 
\textbf{-}Lung volume is the amount of air a lung can hold.\\
\textbf{-}A sample of individuals will undergo a breathing exercise where they are
required to hold their breath for as long as they can take it.\\
\textbf{-}The exercise will be done on an equal number of males and females between the ages of 18 to 30.\\ \\ \\ \\ \\

Thank you. If you have any questions, please contact me at \emph{eugeniamusiimemaria@gmail.com}.

\footnote{\textbf{ MUSIIMEMARIA EUGENIA}}
\end{document} 